\documentclass[10pt,letterpaper]{article}
\usepackage{multicol}
\usepackage{calc}
\usepackage{geometry}
\usepackage{amsmath,amsthm,amsfonts,amssymb}
\usepackage{color,graphicx,overpic}
\usepackage[font=small,labelfont=bf]{caption}
\usepackage{enumitem}

%$$$$$$$$$$$$$ CUSTOM COMMANDS $$$$$$$$$$$$$%
	% Format for creating new command: \newcommand{name}[num][default]{definition} 
\DeclareMathOperator{\di}{d\!} % derivative operator symbol, ex. $\int f(x) \di x$
\newcommand{\Eval}[3]{\left.#1\right\rvert_{#2}^{#3}} % evaluation bar (ex. evaluating integral or $\Eval{F(x)}{0}{2}$)
\newcommand{\barrows}{\textcolor{magenta}{\Longrightarrow}\quad} % long arrow (start of line)
\newcommand{\barrow}{\quad\textcolor{magenta}{\Longrightarrow}\quad} %long arrow 
\newcommand{\sumi}[1][1]{ \sum_{n={#1}}^{\infty} } % sum starting at n = 1
\newcommand{\limi}[1][T]{ \lim_{{#1}\to\infty} } % limit as T goes to infinity
\newcommand{\ddx}[1][x]{ \dfrac{\text{d}}{\di #1} } % derivative operator
\newcommand{\dyd}[2][y]{ \dfrac{\di #1}{\di #2} } % derivative of y w.r.t. x, (ex. \dyd{x} or \dyd[g]{t}).
\newcommand{\bracks}[1]{ \left( #1 \right) } % scaled left and right brackets
\newcommand{\pfrac}[2]{\left(\frac{#1}{#2}\right)} % scaled fraction with left/right brackets
\newcommand{\tpfrac}[2]{\left(\tfrac{#1}{#2}\right)} % tiny fraction with left/right brackets
\newcommand{\abs}[1]{\left| #1 \right|} % scaled fraction with left/right brackets
\newcommand{\norm}[1]{\left\| #1 \right\|} % scaled fraction with left/right brackets
\newcommand{\abracks}[1]{\langle #1 \rangle} % angled l/r brackets for vectors
\newcommand{\cbracks}[1]{\left\{ #1 \right\}} % scaled curly l/r brackets 
\newcommand{\sbracks}[1]{\left[ #1 \right]} % scaled square l/r brackets 

\newcommand{\Z}{\mathbb{Z}}
\newcommand{\Lint}{ \int_{-L}^{L} }
\newcommand{\xabs}{ \abs{x(t)}^2 }
\newcommand{\rect}{ \text{rect} }
\newcommand{\impulse}{ \delta(t) }
\newcommand{\Iint}{ \int_{-\infty}^{\infty} }

\newcommand{\underlineSection}[1][unnamed]{
\subsection*{#1}
\hrule
\vspace{12pt}
}

% Redefine section commands to use less space
\makeatletter
\renewcommand{\section}{\@startsection{section}{1}{0mm}%
                                {-1ex plus -.5ex minus -.2ex}%
                                {0.5ex plus .2ex}%x
                                {\normalfont\large\bfseries}}
\renewcommand{\subsection}{\@startsection{subsection}{2}{0mm}%
                                {-1explus -.5ex minus -.2ex}%
                                {0.5ex plus .2ex}%
                                {\normalfont\normalsize\bfseries}}
\renewcommand{\subsubsection}{\@startsection{subsubsection}{3}{0mm}%
                                {-1ex plus -.5ex minus -.2ex}%
                                {1ex plus .2ex}%
                                {\normalfont\small\bfseries}}
\makeatother
%$$$$$$$$$$$$$$$$$$$$$$$$$$$$$$$$$$$$$$$$$$$%

% Paragraph indentation
\setlength{\parindent}{0cm}
\setlength{\parskip}{0.2cm}

% Page margins
\geometry{
top=1.5cm,
left=1.2cm,
right=1.2cm,
bottom=1.5cm
}

% Line spacing
\renewcommand{\baselinestretch}{1.2} 

\begin{document} 

% Title stuff
\begin{center}
\section*{\Huge{ECE 240 Formula Sheet}}
\hrule
\vspace{6pt}
By Benjamin Kong \& Lora Ma
\end{center}

% Get rid of page numbers
\pagenumbering{gobble}
\footnotesize

\begin{multicols*}{3}

% multicol parameters
% These lengths are set only within the two main columns
\setlength{\columnseprule}{0.5pt}
\setlength{\premulticols}{0.5pt}
\setlength{\postmulticols}{0.5pt}
\setlength{\multicolsep}{0.5pt}
\setlength{\columnsep}{0.5pt}

\underlineSection[1. Time-domain signals]
A continuous-time signal takes the form 
\[ x(t + nT) = x(t) \quad n \in \Z. \]

A signal $z(t) = \alpha x(t + aT_1) + \beta x(t + bT_2)$ will be periodic if
\[ \frac{T_1}{T_2} = \frac{a}{b} \]
for some $a, b \in \Z$.

Let $x(t)$ be some signal.
\begin{itemize}[leftmargin=0.5cm]
\item The \textit{energy} for $t \in (-L, L)$ is given by 
\[ E_{2L} = \Lint \xabs \di t. \]

\item The \textit{total energy} is given by
\[ E = \limi \Lint \xabs \di t. \]

\item The \textit{average power} is given by
\[ P = \limi \dfrac{1}{2L} \Lint \xabs \di t. \]

\item If $x(t)$ is periodic,
\[ P = \dfrac{1}{T} \int_0^T \xabs \di t. \]
\end{itemize}

$E$ finite $\rightarrow$ \textbf{Energy signal} $\rightarrow P = 0$.

$E$ infinite and $P$ finite $\rightarrow$ \textbf{Power signal}.

Periodic signal $\rightarrow$ \textbf{Power signal}.

Let $x(t)$ be some signal.
\begin{itemize}[leftmargin=0.5cm]
\item A \textit{time shift} is represented by
\[ x(t-t_0). \]

\item A \textit{reflection} is represented by
\[ x(-t). \]

\item A signal is \textit{even} if
\[ x(-t) = x(t) \]

\item and \textit{odd} if
\[ x(-t) = -x(t). \]
\end{itemize}

The \textit{unit step signal} is defined as
\[ u(t) = \begin{cases} 
      		1 & t > 0, \\
      		0 & t < 0. 
   		\end{cases}
\]
Note that $u(0) = \frac{1}{2}$.

A \textit{rectangular pulse} is represented as
\[ \rect\tpfrac{t}{T} = u\bracks{t + \tfrac{T}{2}} - u\bracks{t - \tfrac{T}{2}}. \]

A \textit{ramp signal} is represented as
\[ r(t) = tu(t) = \begin{cases} 
		      		t & t \geq 0, \\
		      		0 & t < 0. 
		   		\end{cases}
\]

The \textit{unit impulse} $\impulse$ (Dirac delta function) is defined as
\[ \int_{t_1}^{t_2} x(t) \impulse \di t = x(0) \quad t_1 < 0 < t_2. \]
It has the following properties:
\begin{itemize}[leftmargin=0.5cm]
\item $\impulse = 0$ for $t \neq 0$,
\item $\displaystyle \Iint \impulse \di t = 1$,
\item $\delta(-t) = \impulse$,
\item $\delta(0) = \infty$.
\end{itemize}

\end{multicols*}


\end{document}
